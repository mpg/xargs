% ^^A (utf-8 encoded source)
% 
% \iffalse meta-comment 
%
% Copyright (C) 2007 Manuel P\'egouri\'e-Gonnard <mpg@math.jussieu.fr>
% ----------------------------------------------------------------------
%
% This work may be distributed and/or modified under the
% conditions of the LaTeX Project Public License, either version 1.3c
% of this license or (at your option) any later version.
% The latest version of this license is in:
%
%    http://www.latex-project.org/lppl.txt
%
% and version 1.3c or later is part of all distributions of LaTeX
% version 2006/05/20 or later.
%
% This work has the LPPL maintenance status `maintained'.
%
% The current maintainer of this work is Manuel P\'egouri\'e-Gonnard.
%
% This work consists of the files xargs.dtx and xargs.ins
% and the derived files xargs.sty and xargs.pdf.
%
% \fi
%
% \iffalse ^^A trick below with this \if...
%<*driver>
\ProvidesFile{xargs.dtx}
%</driver>
%<package>\NeedsTeXFormat{LaTeX2e}
%<package>\ProvidesPackage{xargs}
%<*package>
  [2007/10/20 v1.0  extending optional arguments facilities (mpg)]
%</package>
%
%<*driver>
\documentclass[a4paper]{ltxdoc}
\usepackage[utf8]{inputenc}
\usepackage[T1]{fontenc}
%\usepackage{lmodern}
\usepackage[expansion=false]{microtype}
\usepackage[bookmarks=false, colorlinks=true,
  linkcolor=black, urlcolor=black]{hyperref}
\usepackage{fixltx2e}
\newif\iffrenchdoc %\frenchdoctrue % à décommenter pour la doc française
\iffrenchdoc
  \OnlyDescription
  \usepackage[english, french]{babel}
\else
  \usepackage[french, english]{babel}
\fi
\usepackage{xargs}
\newcommand\package{\textsf}
\newcommand\xargs{\package{xargs}}
\setlength\parskip{\smallskipamount}
\begin{document}
  \DocInput{xargs.dtx}
\end{document}
%</driver>
% \fi ^^A double \fi for a single \iffalse because of 
% \fi ^^A the false \if in \newif\iffrenchdoc above :)
%
% \GetFileInfo{xargs.dtx}
% \iffrenchdoc
%   \title{L'extension \xargs}
% \else
%   \title{The \xargs{} package} 
% \fi
% \author{Manuel Pégourié-Gonnard \\ 
%   \href{mailto:mpg@math.jussieu.fr}{mpg@math.jussieu.fr}}
% \date{\fileversion{} (\filedate)}
%
% \maketitle
% \iffrenchdoc \else \tableofcontents \fi
%
%         \section{Introduction}
%
% \iffrenchdoc
%
% \LaTeXe{} permet de définir facilement des commandes ayant un argument
% optionel. Cependant, il y a deux restrictions : il peut y avoir au plus un
% argument optionel, et ce doit être le premier. L'extension \xargs{} fournit
% des versions étendues de \cs{newcommand} et de ses analogues standard, qui ne
% présentent pas ces restrictions.
%
% Vous connaissez peut-être des astuces pour définir des commandes avec
% plusieurs arguments optionels, ou avec l'argument optionel en dernier. Mais
% utiliser ces astuces vous force à résoudre certains problèmes (ordre
% d'utilisation des arguments, gestions des espaces) qui peuvent s'avérer
% délicats. C'est n'est de toute façon pas le genre de choses auxquelles vous
% devriez avoir à réfléchir quand vous rédigez un document.
%
% L'extension \xargs{} vous fournit donc un moyen pratique et (je l'espère)
% robuste de définir de telles commandes, avec une syntaxe intuitive de la forme
% \meta{clé}=\meta{valeur}.
%
% \else
%
% Defining commands with an optional argument is easy in \LaTeXe{}. There is,
% however, two limitations: there can be at most one optional argument and it
% must be the first one. The \xargs{} package provide extended variants of
% \cs{newcommand} \& friends, for which these limitations no longer hold.
%
% You may know some tricks in order to define commands with many optional
% arguments, or with last argument optional, etc. Theses tricks are subject to
% a few problems (using arguments in arbitrary order can be difficult, sometimes
% space is gobbled where it should not), which can be difficult to solve.
% Anyway, you don't want to worry about such tricky things while writing a
% document.
%
% The \xargs{} package provides you with an easy and (hopefully) robust way to
% define such commands, using a nice \meta{key}=\meta{value} syntax.
%
% \fi
%
%         \section{Usage}
%
% \iffrenchdoc
%
% L'extension \xargs{} fournit des analogues de toutes les macros de \LaTeXe{}
% relatives à la définition de macros. Les macros de \xargs{} ont le même nom
% que leur analogue standard, mais avec un |x| supplémentaire à la fin : par
% exemple, \cs{newcommandx}, \cs{renewcommandx} ou encore \cs{newenvironmentx}.
% Elles ont par ailleurs toutes la même syntaxe. Je ne présenterai donc que
% \cs{newcommandx}.
%
% Commençons par un exemple. Après la définition
% \begin{center}
% |\newcommandx*\vect[3][1=1, 3=n]{(#2_{#1},\ldots,#2_{#3})}|
% \end{center}
% ^^A j'en profite pour vérifier les espaces parasites...
% vo\newcommandx*\vect[3][1=1, 3=n]{(#2_{#1},\ldots,#2_{#3})}us
% pouvez utiliser la macro \cs{vect} d'une des façons suivantes :
% \begin{center}
% \begin{tabular}{ll}
% |$\vect{x}$| & $\vect{x}$ \\
% |$\vect[0]{y}$| & $\vect[0]{y}$ \\
% |$\vect{z}[m]$| & $\vect{z}[m]$ \\
% |$\vect[0]{t}[m]$| & $\vect[0]{t}[m]$ \\
% \end{tabular}
% \end{center}
%
% Vous avez sans doute compris qu'il s'agit d'une macro prennant trois arguments
% au total, mais dont le premier et le troisième argument sont optionnels. Ils
% ont chacun une valeur par défaut (respectivement |1| et |n| ici). Vous pouvez
% également remarquer que la syntaxe de \cs{newcomandx} est très proche de celle
% de \cs{newcommand}, la seule différence étant qu'au lieu d'indiquer la valeur
% par défaut de l'unique argument optionel, vous indiquez ici le numéro de
% chaque argument à rendre optionel, suivi de sa valeur par défaut.
%
% Détaillons maintenant la syntaxe de \cs{newcommandx}, qui, répétons-le, est
% aussi la syntaxe de toutes les autres commandes de l'extension \xargs.
% (Pendant qu'on est dans les détails, voici la liste complète de ces commandes
% : \cs{newcommandx} et \cs{renewcommandx} pour les macros simples,
% \cs{providecommandx} pour s'assurer qu'une macro existe,
% \cs{DeclareRobustCommandx} pour (re)définir une macro robuste,
% \cs{CheckCommandx} pour vérifier le sens d'une macro, \cs{newenvironmentx} et
% \cs{renewenvironmentx} pour les environnements.)
%
% \begin{center}
% \cs{newcommandx} \meta{|*|} \marg{commande} \oarg{nombre} \oarg{liste}
% \marg{définition}
% \end{center}
%
% Rappellons brièvement tout ce qui est commun avec la syntaxe usuelle, à savoir
% tout sauf \meta{liste}. Si une |*| est présente, elle signifie que la macro
% crée est \emph{courte} au sens de \TeX{}, c'est-à-dire que ses arguments ne
% peuvent pas contenir de saut de paragraphe (\cs{par} ou ligne vide). La
% \meta{commande} est n'importe quelle séquence de contrôle, que vous pouvez ou
% non entourer d'accolades suivant vos goûts. Le \meta{nombre} définit le nombre
% total d'argument de la \meta{commande}, c'est un entier compris entre $0$ et
% $9$. La \meta{définition} est un texte équilibré en accolades, et où chaque
% caractère |#| est suivi soit d'un chiffre représentant un des arguments, soit
% d'un autre caractère |#|. Les arguments entre crochets sont optionnels.
%
% La partie intéressante maintenant. La \meta{liste} est une\ldots{} liste (!)
% d'éléments de la forme \meta{chiffre}=\meta{valeur}, séparés par des virgules.
% Le \meta{chiffre} doit être un entier compris entre $1$ et le nombre
% d'arguments, donné par \meta{nombre}. La \meta{valeur} est n'importe quelle
% texte équilibré en accolades. Il peut être vide si vous le souhaitez : le
% signe égal qui le précède est alors optionel. Tous les arguments dont le
% numéro figure en tant que \meta{chiffre} dans la \meta{liste} seront
% optionnels, avec pour valeur par défaut celle donnée par la \meta{valeur}
% correspondante.
%
% Concernant l'usage des commandes, notez que, si les arguments $1$ et $2$ (par
% exemple) sont optionnels, vous ne pouvez spécifier de valeur pour l'argument
% $2$ si vous n'en avez pas spécifié pour l'argument $1$. Ce comportement est
% cohérent avec celui de commandes \LaTeX{} existantes, comme \cs{makebox}. Il
% ne provient pas d'une limitation technique mais surtout de mon incapacité à
% imaginer une situation où il est vraiment gênant et une syntaxe intelligente
% pour le contourner.  À ce sujet, observez comment j'ai pris soin de séparer
% les deux arguments optionels par l'argument obligatoire dans le définition de
% \cs{vect} ci-dessus.
%
% Quelques remarques supplémentaires sur la syntaxe de la \meta{liste}, que vous
% pouvez sauter si vous êtes familiers avec la syntaxe fournie par
% \package{xkeyval}. Vu que les éléments sont séparés par des virgules, si une
% \meta{valeur} doit contenir une virgule, il faut entourer la valeur par des
% accolades pour protéger la virgule. (Cette précaution est également
% indispensable si la \meta{valeur} contient une accolade fermante.) Ne vous
% inquiétez pas, cette parie d'accolade sera retirée ultérieurement. D'ailleurs,
% jusqu'à trois paires d'accolades seront retirées ainsi, et si vous voulez
% vraiment que votre valeur reste entourée d'acollades, il vous faudra écrire
% quelque chose comme~|1={{{{\large blabla}}}}|.
%
% La dernière particularité de la \meta{liste} que vous devez connaître, est
% qu'elle ne doit pas contenir de |\par| (ou de ligne vide). C'est le seul point
% (à ma connaissance) sur lequel les macros d'\xargs{} diffèrent des macros
% standard. Cette limitation est liée à un choix dans l'implémentation
% d'\package{xkeyval}, que j'ai eu un peu la flemme de contourner (en fait, je
% ne suis pas pleinement satisfait de la fiabilité des contournements que je
% connais). Vous pouvez utiliser \cs{endgraf} à la place de \cs{par} en cas de
% besoin. Si cette limitation vous gêne, merci de me le faire savoir.
%
% Parlons maintenant de trois fonctionnalités d'\xargs{} qui sont plus ou
% moins cachées mais que j'espère pratiques. La première est que les macros sont
% crées si possible de façon économique : si vous utilisez \cs{newcommandx} pour
% définir un macro que vous auriez pu définir avec \cs{newcommand}, \xargs{} le
% détectera et utilisera automatiquement \cs{newcommand} à la place. Ainsi, vous
% n'avez à vous soucier de rien et vous pouvez utiliser tout le temps
% \cs{newcommandx} sans vous poser de questions.
%
% La deuxième fonctionnalité est elle aussi transparente, mais comme j'ai pris
% la peine de la coder, il faut bien que j'en parle. Une macro crée par \xargs{}
% n'avalera pas les espaces qui la suivent, même si le dernier argument est
% optionel et n'est pas spécifié. C'est un des avantages d'\xargs{} sur les
% astuces classiques. Comparez les deux sur l'exemple (idiot) suivant :
% \begin{quote}
% |\newcommand\compliment[1]{#1 est gentil\complinterne}|\\
% |\newcommand\complinterne[1][le]{#1}|\\
% |\newcommandx\complimentx[2][2=le]{#1 est gentil#2}|
% \end{quote}
% \newcommand\compliment[1]{#1 est gentil\complinterne}
% \newcommand\complinterne[1][le]{#1}
% \newcommandx\complimentx[2][2=le]{#1 est gentil#2}
% \begin{center}\begin{tabular}{ll}
% |\compliment{Clothilde} et\ldots| & 
% \compliment{Clothilde} et\ldots   \\ 
% |\complimentx{Clothilde} et\ldots| & 
% \complimentx{Clothilde} et\ldots  \\ 
% \end{tabular}\end{center}
% Observez comme l'espace a été avalé de façon indésirable dans le premier cas.
%
% Enfin, les macros d'\xargs{} essayent de se comporter en tous points comme
% leurs homoloques standard. Il y a deux exceptions : la première est que vous
% ne vous pouvez pas utiliser \cs{par} comme vous voulez, je l'ai dit plus haut,
% et je le regrette. La deuxième réside dans certains comportements de
% \cs{CheckCommandx}. En effet, à l'heure où j'écris ces lignes,
% \cs{CheckCommand} souffre de deux bugs (voir
% \href{http://www.latex-project.org/cgi-bin/ltxbugs2html?pr=latex/3971&introduction=yes&state=open}
% {PR/3971}) que j'espère avoir évités dans \cs{CheckCommandx}.
% \bigskip
%
% Vous savez maintenant tout ce qu'il y a à savoir sur l'utilisation de
% \xargs{}. Si vous souhaitez vous pencher sur son implémentation, il vous
% faudra lire les commentaires en anglais car je n'ai pas eu le courage de
% commenter mon code en deux langues. Si vous découvrez des bugs, merci de me
% les signaler. Si vous trouvez cette extension utile, ça me ferait aussi
% plaisir de le savoir.
%
% \begin{center}\large
% C'est tout pour cette fois ! \\
% Amusez-vous bien avec \LaTeX{} !
% \end{center}
%
% \else
%
% The \xargs{} package defines an extended variant for every \LaTeX{} macro
% related to macro definition. \xargs{}'s macro are named after their \LaTeX{}
% couterparts, just adding an |x| at end, e.g. \cs{newcommandx},
% \cs{renewcommandx} or \cs{newenvironmentx}. Since they all share the same
% syntax (closely ressembling \cs{newcommand}'s one), I will only explain with
% \cs{newcommandx}.
%
% Let's begin with the following example.
% \begin{center}
% |\newcommandx*\vect[3][1=1, 3=n]{(#2_{#1},\ldots,#2_{#3})}|
% \end{center}
% \newcommandx*\vect[3][1=1, 3=n]{(#2_{#1},\ldots,#2_{#3})}
% \begin{center}
% \begin{tabular}{ll}
% |$\vect{x}$| & $\vect{x}$ \\
% |$\vect[0]{y}$| & $\vect[0]{y}$ \\
% |$\vect{z}[m]$| & $\vect{z}[m]$ \\
% |$\vect[0]{t}[m]$| & $\vect[0]{t}[m]$ \\
% \end{tabular}
% \end{center}
%
% You surely understood \cs{vect} is now a macro with 3 arguments, the first and
% third being optional. They both have their own default value (resp.\@ |1| and
% |n|). You maybe noticed how \cs{newcommandx}'s syntax closely ressembles
% \cs{newcommand}'s syntax: The only difference is, instead of the default value
% of the only optional argument, you have to specifiy the number of the optional
% arguments, followed by a = and their default value.
%
% Now let's have a closer look at \cs{newcommandx}'s syntax, shared by all other
% \xargs{} commands. (While dealing with details, here is the complete list of
% those: \cs{newcommandx} and \cs{renewcommandx} for simple macros,
% \cs{providecommandx} to make sure a macro exists, \cs{DeclareRobustCommandx}
% to (re)define a robust macro, \cs{CheckCommandx} to check if a macro has the
% correct meaning, \cs{newenvironmentx} and \cs{renewenvironmentx} for
% environments.)
%
% \begin{center}
% \cs{newcommandx} \meta{|*|} \marg{command} \oarg{number} \oarg{list}
% \marg{definition}
% \end{center}
%
% Everything here is the same as usual \cs{newcommand} syntax, except
% \meta{list}. Let's recall this briefly. The optional |*| make \LaTeX{} define
% a ``short'' macro, that is a macro that won't accept a paragraph break
% (\cs{par} or an empty line) inside its argument, otherwise the macro will be
% long.  \meta{command} is any control sequence, and can but need not be
% enclosed in braces, as you like. The \meta{number} specifies how many
% arguments your macro will take (including optional ones): It should be a
% non-negative integer, and at most $9$. The macro's \meta{definition} is a
% balanced text, where every |#| sign must be followed with a number, thus
% representing an argument, or with another |#| sign. The two arguments
% \meta{number} and \meta{list} are optionals.
%
% Now comes the new and funny part. \meta{list} is a coma-separated list of
% element \meta{digit}=\meta{value}. Here, \meta{digit} should be non-zero, and
% at most \meta{number} (the total number of arguments). The \meta{value} is any
% balanced text, and can be empty. If so, the |=| sign becomes optional: You
% only need to write \meta{digit} if you want the \meta{digit}th argument to be
% optional, with empty default value. Of course, every argument whose number is
% a \meta{digit} in the \meta{list} becomes optional, with \meta{value} as its
% default value.
%
% While using a macro with many optional arguments, keep in mind the following
% fact. If arguments, say $1$ and $2$, are optional, then if you specify a value
% for only one optional argument, it will be used for argument $1$, and argument
% $2$ will be considered non-specified (thus its default value will be used).
% This behaviour is consistent with existing \LaTeX{}'s command, like
% \cs{makebox}. It isn't a technical limitation, I just couldn't imagine a
% better way to do. By the way, please notice the way I separated the two
% optional arguments from the exemple above in order to be able to use them
% independantly.
%
% If you are not very familiar with some aspects of the syntax provided by the
% \package{xkeyval} package, you may be interested in the following remarks
% about the syntax of \meta{list}. Since \meta{list} is coma-separated, if you
% want to use a coma inside a \meta{value}, you need to enclose it (either the
% coma or the whole \meta{value}) in braces. The same applies if you want to use
% a closing square bracket inside the \meta{list}. Don't worry about those
% unwanted braces, they will be removed later. Actually, \package{xkeyval}
% removes up to $3$ braces set: If you really want braces around a value, you
% need to type something like |1={{{{\large stuff}}}}|.
%
% The last thing you (maybe) need to know about \meta{list} is a little
% limitation of \xargs{}, inherited from \package{xkeyval}, which I didn't
% managed to work around (actually, I know a way to do it, but it fails in rare
% cases involving \cs{let}ing an active character equal to a brace, so I decided
% not to include it). So what is this problem? It's just you cannot use \cs{par}
% (or an empty line) in the \meta{list}. If you need a paragraph break in a
% \meta{value}, try \cs{endgraf}. If this issue really bothers you, please let
% me know.
%
% There's only three features or \xargs{} not yet discussed. Since they are all
% ``good'' features, you may not need to read the end of the documentation, but
% since I made the effort to implement it, I want to talk about it. First one is
% that macros are made in a minimalistic way: If you use \cs{newcommandx} to
% create a macro you could have made with \cs{newcommand}, \xargs{} will notice
% and use \cs{newcommand} internally. So, you can allways use \cs{newcommandx}
% without bothering.
%
% Second feature consist in avoiding a possible problem with spaces after a
% macro whose last argument is optional. If defined in a naive way, such macros
% would gobble spaces after them when the optional argument is not specified,
% but macros created with \xargs{} don't, so you don't need to take any special
% care about spaces.
%
% Last, \xargs{} macros try to behave exactly as standard \LaTeX{} macros do. As
% far as I know, there are only two exceptions. I allready mentionned the first,
% which is the problem with \cs{par} in default value, due to \package{xkeyval}.
% The other one is that, right now, the current implementation of
% \cs{CheckCommand} has some problems (see
% \href{http://www.latex-project.org/cgi-bin/ltxbugs2html?pr=latex/3971&introduction=yes&state=open}
% {PR/3971}). I tried not to reproduce them in \cs{CheckCommandx}.
%
% \fi
%
% \StopEventually{}
%
%         \section{Implementation}
%
% If you are familiar with the way \LaTeXe{} manages macros with an optional
% argument, then you will probably not be surprised by the \xargs{} way.
% Indeed, with \LaTeXe{}, a command \cs{foo} defined with, say
% |\newcommand*\foo[2][bar]{baz}| is implemented as the pair:
% \begin{quote}
% |\foo=macro:->\@protected@testopt\foo\\foo{bar}| \\
% |\\foo=macro:[#1]#2->baz|
% \end{quote}
% With \xargs's \cs{newcommandx}, a macro \cs{vect} as above is
% implemented as:
% \begin{quote}
% |\vect=macro:->\@protected@testopt@xargs\vect\\vect|\\
% \hspace*{1em}|{\xargs@test@opt{0},\xargs@put@arg,\xargs@test@opt{n},}|\\
% |\\vect=macro:[#1]#2[#3]->(#2_{#1},\ldots ,#2_{#3})|
% \end{quote}
%
% This is very similar: the ``real'' macro is the one whose name begins with a
% \cs{} and the macro called by the user just checks the protection context and
% collects the arguments for the internal macro, using the default value if none
% is given for the optional argument(s). However, the analogy ends here, since
% in ``normal'' \LaTeX{} there is only one optional argument, but
% \xargs{} commands need more information about optional arguments,
% namely their position, and not only the default values.
%
% This information is stored as a coma-separated list of ``actions'', each
% action consisting of either the single command \cs{xargs@put@arg}, which
% denotes a mandatory argument and makes \LaTeX{} grab it and add it to the list
% of arguments to be passed to the internal macro, or \cs{xargs@test@opt} with
% argument the default value, which denotes an optional argument. In the later
% case, the presence of the optional argument is checked in a way slightly
% differing from \LaTeXe's \cs{@ifnextchar}, then the relevant value added to
% the arguments list.
%
% All this argument grabing job is done with a loop that read and executes each
% action from the originating list, and concurrently builds an argument list
% such as |[0]{x}[m]| to be passed to |\\vect| for example. The first part of
% the code consists of those macros used for the argument grabbing and execution
% of internal command process.
%
% \bigskip
% First, load the \package{xkeyval} package for it's nice key=value syntax.
%    \begin{macrocode}
\RequirePackage{xkeyval}
%    \end{macrocode}
%
% \begin{macro}{\xargs@max}
% \begin{macro}{\xargs@temp}
% \begin{macro}{\xargs@toksa}
% \begin{macro}{\xargs@toksb}
% Then allocate a few registers and make sure the name of our private scratch
% macro is free for use. Note that for certain uses, we really need a \cs{toks}
% register because the string used can possibly contain |#| characters.
% Sometimes I also use a \cs{toks} register instead of a macro just for ease of
% use (writing less \cs{expandafter}s).
%    \begin{macrocode}
\@ifdefinable\xargs@max{\newcount\xargs@max}
%    \end{macrocode}
%    \begin{macrocode}
\@ifdefinable\xargs@temp\relax
\@ifdefinable\xargs@toksa{\newtoks\xargs@toksa}
\@ifdefinable\xargs@toksb{\newtoks\xargs@toksb}
%    \end{macrocode}
% \end{macro}
% \end{macro}
% \end{macro}
% \end{macro}
% \begin{macro}{\@protected@testop@xargs}
% This first macros closely resembles kernel's \cs{@protected@testopt}
% (similarity in their names is intentional, see \cs{CheckCommandx}). It just
% checks the protection context and call the real argument grabbing macro.
%    \begin{macrocode}
\newcommand*\@protected@testopt@xargs[1]{%
  \ifx\protect\@typeset@protect
    \expandafter\xargs@read
  \else
    \@x@protect#1%
  \fi}
%    \end{macrocode}
% \end{macro}
% \begin{macro}{\xargs@read}
% Initiate the loop. \cs{xargs@toksa} will become the call to the internal macro
% with all arguments, \cs{xargs@toksb} contains the actions list for arguments
% grabbing. 
%    \begin{macrocode}
\newcommand*\xargs@read[2]{%
  \xargs@toksa{#1}%
  \xargs@toksb{#2}%
  \xargs@continue}
%    \end{macrocode}
% \end{macro}
% \begin{macro}{\xargs@continue}
% \begin{macro}{\xargs@pick@next}
% Each iteration of the loop consist of two steps: pick the next action (and
% remove it from the list), and execute it. When there is no more action in the
% list, it means the arguments grabbing stage is over, and it's time to execute
% the internal macro by expanding then contents of \cs{xargs@toksa}.
%    \begin{macrocode}
\newcommand\xargs@continue{%
  \expandafter\xargs@pick@next\the\xargs@toksb,\@nil
  \xargs@temp}
%    \end{macrocode}
%    \begin{macrocode}
\@ifdefinable\xargs@pick@next{%
  \def\xargs@pick@next#1,#2\@nil{%
    \def\xargs@temp{#1}%
    \xargs@toksb{#2}%
    \ifx\xargs@temp\empty
      \def\xargs@temp{\the\xargs@toksa}%
    \fi}}
%    \end{macrocode}
% \end{macro}
% \end{macro}
% \begin{macro}{\xargz@put@arg}
% \begin{macro}{\xargz@test@opt}
% \begin{macro}{\xargz@put@opt}
% Now have a look at the argument grabbing macros. The first one,
% \cs{xargs@put@arg}, just reads an undelimited argument in the input stack and
% add it to the arguments list. \cs{xargs@testopt} checks if the next non-space
% token is a square bracket to decide if it have to read an argument from the
% input or use the default value, and takes care to enclose it in square
% brackets.
%    \begin{macrocode}
\newcommand\xargs@put@arg[1]{%
  \xargs@toksa\expandafter{\the\xargs@toksa{#1}}%
  \xargs@continue}
%    \end{macrocode}
%    \begin{macrocode}
\newcommand*\xargs@test@opt[1]{%
  \xargs@ifnextchar[%]
    {\xargs@put@opt}%
    {\xargs@put@opt[{#1}]}}
%    \end{macrocode}
%    \begin{macrocode}
\@ifdefinable\xargs@put@opt{%
  \long\def\xargs@put@opt[#1]{%
    \xargs@toksa\expandafter{\the\xargs@toksa[{#1}]}%
  \xargs@continue}}
%    \end{macrocode}
% \end{macro}
% \end{macro}
% \end{macro}
% \begin{macro}{\xargs@ifnextchar}
% \begin{macro}{\xargs@ifnch}
% \begin{macro}{\xargs@xifnch}
% You probably noticed that \cs{xargs@testopt} doesn't use kernel's
% \cs{@ifnextchar}. The reason is, I don't want macros to gobble space if their
% last argument is optional and not specified. Indeed, it would be strange to
% have spaces after |\vect[0]{x}| gobbled. So the modified version of
% \cs{@ifnextchar} below works like kernel's one, except that it remembers how
% many spaces it gobbles and restitutes them in case the next non-space
% character isn't a match.
%    \begin{macrocode}
\newcommand\xargs@ifnextchar[3]{%
  \let\xargs@temp\empty
  \let\reserved@d=#1%
  \def\reserved@a{#2}%
  \def\reserved@b{#3}%
  \futurelet\@let@token\xargs@ifnch}
%    \end{macrocode}
%    \begin{macrocode}
\newcommand\xargs@ifnch{%
  \ifx\@let@token\@sptoken
    \edef\xargs@temp{\xargs@temp\space}%
    \let\reserved@c\xargs@xifnch
  \else
    \ifx\@let@token\reserved@d
      \let\reserved@c\reserved@a
    \else
      \def\reserved@c{\expandafter\reserved@b\xargs@temp}%
    \fi
  \fi
  \reserved@c}
%    \end{macrocode}
%    \begin{macrocode}
\@ifdefinable\xargs@xifnch{%
  \expandafter\def\expandafter\xargs@xifnch\space{%
    \futurelet\@let@token\xargs@ifnch}}
%    \end{macrocode}
% \end{macro}
% \end{macro}
% \end{macro}
% \bigskip
% 
% Okay, we're finished with the execution related macros. Now let's start with
% stuff for the definition of macros. The idea is to collect through
% \package{xkeyval} at most 9 actions numbered 1 to \cs{xargs@max} (the total
% number of arguments) of the type seen above, then structure them in a
% coma-separated list for use in the user macro's definition. Special care is
% taken to define simpler macros in the two special cases where all arguments,
% possibly except the first one, are mandatory (standard \LaTeXe{} cases).
%
% \begin{macro}{\@namenewc} \begin{macro}{\xargs@action@1}
% \begin{macro}{\xargs@action@2} \begin{macro}{\xargs@action@3}
% \begin{macro}{\xargs@action@4} \begin{macro}{\xargs@action@5}
% \begin{macro}{\xargs@action@6} \begin{macro}{\xargs@action@7}
% \begin{macro}{\xargs@action@8} \begin{macro}{\xargs@action@9}
% So our first task is to define container macros for these at most nine
% actions, with default value \cs{xargs@put@arg} since every argument is
% mandatory unless specified.
%    \begin{macrocode}
\providecommand\@namenewc[1]{%
  \expandafter\newcommand\csname #1\endcsname}
%    \end{macrocode}
%    \begin{macrocode}
\@namenewc{xargs@action@1}{\xargs@put@arg}
\@namenewc{xargs@action@2}{\xargs@put@arg}
\@namenewc{xargs@action@3}{\xargs@put@arg}
\@namenewc{xargs@action@4}{\xargs@put@arg}
\@namenewc{xargs@action@5}{\xargs@put@arg}
\@namenewc{xargs@action@6}{\xargs@put@arg}
\@namenewc{xargs@action@7}{\xargs@put@arg}
\@namenewc{xargs@action@8}{\xargs@put@arg}
\@namenewc{xargs@action@9}{\xargs@put@arg}
%    \end{macrocode}
% \end{macro} \end{macro}
% \end{macro} \end{macro}
% \end{macro} \end{macro}
% \end{macro} \end{macro}
% \end{macro} \end{macro}
% \begin{macro}{\xargs@def@key}
% The next macro will define key for us. Its first argument is the key's
% number. The second argument will be discussed later.
%    \begin{macrocode}
\newcommand*\xargs@def@key[2]{%
  \define@key[xargs]{key}{#1}[]{%
%    \end{macrocode}
% The first thing do to, before setting any action, is to check wether this key
% can be used for this command, and complain if not.
%    \begin{macrocode}
    \ifnum\xargs@max<#1
      \PackageError{xargs}{%
        Illegal argument label in\MessageBreak
        optional arguments description%
        }{%
        You are trying to make optional an argument whose label (#1)
        \MessageBreak is higher than the total number (\the\xargs@max)
        of parameters. \MessageBreak This can't be done and your demand
        will be ignored.}%
    \else
%    \end{macrocode}
% If the key number is correct, it may be that the user is trying to use it
% twice for the same command. Since it's probably a mistake, issue a warning in
% such case.
%    \begin{macrocode}
      \expandafter\expandafter\expandafter
      \ifx\csname xargs@action@#1\endcsname\xargs@put@arg \else
        \PackageWarning{xargs}{%
          Argument #1 was allready given a default value.\MessageBreak
          Previous value will be overriden.\MessageBreak}%
      \fi
%    \end{macrocode}
% If everything looks okay, define the action to be \cs{xargs@test@opt} with the
% given value, and execute the (for now) mysterious second argument.
%    \begin{macrocode}
      \@namedef{xargs@action@#1}{\xargs@test@opt{##1}}%
      #2%
    \fi}}
%    \end{macrocode}
% \end{macro}
% \begin{macro}{\ifxargs@firstopt@}
% \begin{macro}{\ifxargs@otheropt@}
% \begin{macro}{\xargs@key@1}
% \begin{macro}{\xargs@key@2} \begin{macro}{\xargs@key@3}
% \begin{macro}{\xargs@key@4} \begin{macro}{\xargs@key@5}
% \begin{macro}{\xargs@key@6} \begin{macro}{\xargs@key@7}
% \begin{macro}{\xargs@key@8} \begin{macro}{\xargs@key@9}
% The second argument just consist in setting the value for some \cs{if} wich
% will keep track of the existence of an optional argument other than the first
% one, and the of the possibly otpional nature of the first. Such information
% will be usefull when we will have to decide if we use the \LaTeXe{} standard
% way or \xargs{} custom one to define the macro.
%    \begin{macrocode}
\newif\ifxargs@firstopt@
\newif\ifxargs@otheropt@
%    \end{macrocode}
% Now actually define the keys. 
%    \begin{macrocode}
\xargs@def@key1\xargs@firstopt@true
\xargs@def@key2\xargs@otheropt@true \xargs@def@key3\xargs@otheropt@true
\xargs@def@key4\xargs@otheropt@true \xargs@def@key5\xargs@otheropt@true
\xargs@def@key6\xargs@otheropt@true \xargs@def@key7\xargs@otheropt@true
\xargs@def@key8\xargs@otheropt@true \xargs@def@key9\xargs@otheropt@true
%    \end{macrocode}
% \end{macro} \end{macro} \end{macro}
% \end{macro} \end{macro} \end{macro}
% \end{macro} \end{macro} \end{macro}
% \end{macro} \end{macro}
% \begin{macro}{\xargs@setkeys}
% \begin{macro}{\xargs@check@keys}
% We set the keys with the starred version of \cs{setkeys}, so we can check if
% there were some strange keys we cannot handle, and issue a meaningfull warning
% if there are some.
%    \begin{macrocode}
\newcommand\xargs@setkeys[1]{%
  \setkeys*[xargs]{key}{#1}%
  \xargs@check@keys}
%    \end{macrocode}
%    \begin{macrocode}
\newcommand\xargs@check@keys{%
  \ifx\XKV@rm\empty \else
    \xargs@toksa\expandafter{\XKV@rm}%
    \PackageError{xargs}{%
      Illegal argument label in\MessageBreak
      optional arguments description%
      }{%
      You can only use non-zero digits as argument labels.\MessageBreak
      You wrote: "\the\xargs@toksa".\MessageBreak
      I can't understand this and I'm going to ignore it.}%
  \fi}
%    \end{macrocode}
% \end{macro}
% \end{macro}
% \begin{macro}{\xargs@add@args}
% Now our goal is to build two lists from our up to nine actions macros. The
% first is the coma-separated list of actions allready discussed. The second
% is the parameter text for use  in the definition on the internal macro, for
% example |[#1]#2[#3]|. The next macro takes the content of a
% \cs{xargs@action@X} macro for argument and adds the corresponding items to
% this lists. It checks if the first token of its parameter is \cs{xargs@testopt}
% in order to know if the |#n| has to be enclosed in square brackets.
%    \begin{macrocode}
\newcommand\xargs@add@args[1]{%
  \xargs@toksa\expandafter{\the\xargs@toksa #1,}%
  \expandafter
  \ifx\@car#1\@nil\xargs@put@arg
    \xargs@toksb\expandafter\expandafter\expandafter{%
      \the\expandafter\xargs@toksb\expandafter##\the\@tempcnta}%
  \else
    \xargs@toksb\expandafter\expandafter\expandafter{%
      \the\expandafter\xargs@toksb\expandafter
      [\expandafter##\the\@tempcnta]}%
  \fi}
%    \end{macrocode}
% \end{macro}
% \begin{macro}{\xargs@process@keys}
% Here comes the main input processing macro, which prepares the information needed
% to define the final macro, and expands it to the defining macro.
%    \begin{macrocode}
\@ifdefinable\xargs@process@keys{%
  \long\def\xargs@process@keys#1[#2]{%
%    \end{macrocode}
% Some initialisations. We work inside a group so that the default values for
% the \cs{xargs@action@X} macros and the \cs{xargs@XXXopt@} be automatically
% restored for the next time.
%    \begin{macrocode}
  \begingroup
  \xargs@setkeys{#2}%
  \xargs@toksa{}\xargs@toksb{}%
%    \end{macrocode}
% Then the main loop actually builds up the two lists in the correct order.
%    \begin{macrocode}
  \@tempcnta0
  \@whilenum \xargs@max>\@tempcnta \do{%
    \advance\@tempcnta1
    \expandafter\expandafter\expandafter\xargs@add@args
    \expandafter\expandafter\expandafter{%
      \csname xargs@action@\the\@tempcnta\endcsname}}%
%    \end{macrocode}
% Now we need to  address a special case: If only the first argument is
% optional, we use \LaTeXe's standard \cs{newcommand}, and we dont need an
% actions list like the one just build, but only the default value for the first
% argument. In this case, we extract this value from \cs{xargs@action@1} by
% expanding it three times with a modified \cs{xargs@testopt}.
%    \begin{macrocode}
  \ifxargs@otheropt@ \else
    \ifxargs@firstopt@
      \let\xargs@test@opt\@firstofone
      \xargs@toksa\expandafter\expandafter\expandafter
      \expandafter\expandafter\expandafter\expandafter{%
        \csname xargs@action@1\endcsname}
    \fi
  \fi
%    \end{macrocode}
% Finally expand the stuff to the next macro. 
%    \begin{macrocode}
  \expandafter\expandafter\expandafter\xargs@choose@def
  \expandafter\expandafter\expandafter#1%
  \expandafter\expandafter\the\xargs@max
  \expandafter{\the\xargs@toksa}}}
%    \end{macrocode}
% \end{macro}
% \begin{macro}{\xargs@choose@def}
% It's time to make a choice about the method used to define the macro,
% depending of the number and place of its optional arguments. Two cases are
% handled by \LaTeX, the last is \xargs{} non-trivial case. Note the
% \cs{expandafter}\cs{endgroup} trick which allows us to use the \cs{if}s
% outside the group but have them restored to false just after.
%    \begin{macrocode}
\newcommand\xargs@choose@def[4]{%
  \expandafter\expandafter\expandafter
  \endgroup
  \ifxargs@otheropt@
    \expandafter\xargs@def@cmd\expandafter#1\expandafter{%
      \the\xargs@toksb}{#3}{#4}%
  \else
    \ifxargs@firstopt@
      \@xargdef#1[#2][#3]{#4}%
    \else
      \@argdef#1[#2]{#4}%
    \fi
  \fi}
%    \end{macrocode}
% \end{macro}
% \begin{macro}{\xargs@def@cmd}
% This is \xargs{} variant of \cs{xargdef}. The only thing not yet
% expanded is the internal macro name, for which we use all the
% \cs{expandafter}s.
%    \begin{macrocode}
\newcommand\xargs@def@cmd[4]{%
  \@ifdefinable#1{%
    \expandafter\def\expandafter#1\expandafter{%
      \expandafter\@protected@testopt@xargs
      \expandafter#1\csname \string#1\endcsname{#3}}%
    \l@ngrel@x\expandafter\def\csname \string#1\endcsname#2{#4}}}
%    \end{macrocode}
% \end{macro}
% \bigskip
% \begin{macro}{\newcommandx}
% \begin{macro}{\xargs@newc}
% All the (tricky?) internal macros are ready. It's time to define the user
% commands, beginning with \cs{newcommandx}. Like it's standard version, it
% just checks the star and call the next macro wich grabs the number of
% arguments.
%    \begin{macrocode}
\newcommand\newcommandx{%
  \@star@or@long\xargs@newc}
%    \end{macrocode}
%    \begin{macrocode}
\newcommand*\xargs@newc[1]{%
  \@testopt{\xargs@set@max{#1}}{0}}
%    \end{macrocode}
% \end{macro}
% \end{macro}
% \begin{macro}{\xargs@set@max}
% Set the value of \cs{xargs@max}. If no optional arguments description
% follows, simply call \cs{argdef} because all the complicated stuff is useless
% here.
%    \begin{macrocode}
\@ifdefinable\xargs@set@max{%
  \def\xargs@set@max#1[#2]{%
    \kernel@ifnextchar[%]
      {\xargs@max=#2 \xargs@check@max{#1}}%
      {\@argdef#1[#2]}}}
%    \end{macrocode}
% \end{macro}
% \begin{macro}{\xargs@check@max}
% To avoid possible problems later, check right now that \cs{xargs@max} value is
% valid. If not, warn the user and treat this value as zero. Then begin the key
% processing.
%    \begin{macrocode}
\newcommand\xargs@check@max{%
  \ifcase\xargs@max \or\or\or\or\or\or\or\or\or\else
    \PackageError{xargs}{Illegal number, treated as zero}{The total
      number of arguments must be in the 0..9 range.\MessageBreak
      Since your value is illegal, i'm going to use 0 instead.}
    \xargs@max0
  \fi
  \xargs@process@keys}
%    \end{macrocode}
% \end{macro}
% The other macros (\cs{renewcommand} etc) closely resemble their kernel
% couterpart, since they are mostly wrappers around some call to
% \cs{xargs@newc}. There is however an exception, \cs{CheckCommand}, which I
% will treat first. Here my way differs from the kernel's one, since current
% implemetation of \cs{CheckCommand} in the kernel suffers from two bugs (see
% PR/3971).
% \begin{macro}{\CheckCommandx}
% We begin as usual detecting the possible star.
%    \begin{macrocode}
\@ifdefinable\CheckCommandx{%
  \def\CheckCommandx{%
    \@star@or@long\xargs@CheckC}}
\@onlypreamble\CheckCommandx
%    \end{macrocode}
% \end{macro}
% \begin{macro}{\xargs@CheckC}
% \begin{macro}{\xargs@check@a}
% \begin{macro}{\xargs@check@b}
% First, we don't use the |#2#{| trick from the kernel, since it can fail if there
% are braces in the default values. Instead, we follow the argument grabing
% method used for \cs{new@environment}, ie calling \cs{kernel@ifnextchar}
% explicitly.
%    \begin{macrocode}
\@ifdefinable\xargs@CheckC{%
  \def\xargs@CheckC#1{%
    \@testopt{\xargs@check@a#1}0}}
\@onlypreamble\xargs@CheckC
%    \end{macrocode}
%    \begin{macrocode}
\@ifdefinable\xargs@check@a{%
  \def\xargs@check@a#1[#2]{%
    \kernel@ifnextchar[%]
      {\xargs@check@b#1[#2]}%
      {\xargs@check@c#1{[#2]}}}}
\@onlypreamble\xargs@check@a
%    \end{macrocode}
%    \begin{macrocode}
\@ifdefinable\xargs@chech@b{%
  \def\xargs@check@b#1[#2][#3]{%
    \xargs@check@c{#1}{[#2][{#3}]}}}
\@onlypreamble\xargs@check@b
%    \end{macrocode}
% \end{macro}
% \end{macro}
% \end{macro}
% \begin{macro}{\xargs@CheckC}
% Here come the major difference with the kernel version. If |\\reserved@a| is
% defined, we not only check that it is equal to |\\foo| (assuming \cs{foo} is
% the macro being tested), we also check that \cs{foo} makes something sensible
% by calling \cs{xargs@check@d}.
%    \begin{macrocode}
\newcommand\xargs@check@c[3]{%
  \xargs@toksa{#1}%
  \expandafter\let\csname\string\reserved@a\endcsname\relax
  \xargs@renewc\reserved@a#2{#3}%
  \@ifundefined{\string\reserved@a}{%
    \ifx#1\reserved@a \else
      \xargs@check@complain
    \fi
    }{%
    \expandafter
    \ifx\csname\string#1\expandafter\endcsname
        \csname\string\reserved@a\endcsname
      \begingroup\escapechar 92
      \xargs@check@d
    \else
      \xargs@check@complain
    \fi}}
\@onlypreamble\xargs@check@c
%    \end{macrocode}
% \end{macro}
% So, what do we want \cs{foo} to do? If |\\foo| is defined, \cs{foo} should
% begin with one of the followings:
% \begin{quote}
% |\@protected@testopt \foo \\foo| \\
% |\@protected@testopt@xargs \foo \\foo|
% \end{quote}
% Since I'm to lazy to really check this, the \cs{xargs@check@d} macro only
% checks if the \cs{meaning} of \cs{foo} begins with \cs{@protected@test@opt}
% (without a space after it). It does this using a macro with delimited
% argument. Here are preliminaries to this definition: We need to have this
% string in \cs{catcode} 12 tokens.
%    \begin{macrocode}
\def\xargs@temp{\@protected@testopt}
{ \escapechar 92
  \global\xargs@toksa\expandafter{\meaning\xargs@temp}}
\def\xargs@temp#1 \@nil{\def\xargs@temp{#1}}
\expandafter\xargs@temp\the\xargs@toksa\@nil
%    \end{macrocode}
% \begin{macro}{\xargs@check@d}
% \begin{macro}{\xargs@check@e}
% Now, \cs{xargs@check@c} just pass the \cs{meaning} of the command \cs{foo} being
% checked to the allready mentionned macro with delimited arguments, which will
% check if its first argument is empty (ie, if \cs{foo}'s \cs{meaning} starts
% with what we want) and complain otherwise.
%    \begin{macrocode}
\@ifdefinable\xargs@check@d{%
  \expandafter\newcommand\expandafter\xargs@check@d\expandafter{%
    \expandafter\expandafter\expandafter\xargs@check@e
    \expandafter\meaning\expandafter\reserved@a\xargs@temp\@nil}}
\@onlypreamble\xargs@check@d
%    \end{macrocode}
%    \begin{macrocode}
\@ifdefinable\xargs@check@e{%
  \expandafter\def\expandafter\xargs@check@e
  \expandafter#\expandafter1\xargs@temp#2\@nil{%
    \endgroup
    \ifx\empty#1\empty \else
      \xargs@check@complain
    \fi}}
\@onlypreamble\xargs@check@e
%    \end{macrocode}
% \end{macro}
% \end{macro}
% \begin{macro}{\xargs@check@complain}
% The complaining macro uses the name saved by \cs{xargs@check@c} in
% \cs{xargs@toksa} in order to complain about the correct macro.
%    \begin{macrocode}
\newcommand\xargs@check@complain{%
  \PackageWarningNoLine{xargs}{Command \the\xargs@toksa has changed.
    \MessageBreak Check if current package is valid}}
\@onlypreamble\xargs@check@complain
%    \end{macrocode}
% \end{macro}
% \bigskip
%
% From now on, there is absolutely nothing to comment on, since the next macros
% are mainly wrappers around \cs{xargs@newc}, just as kernel's ones are wrappers
% around \cs{new@command}. So the code below is only copy/paste with
% search\&replace from the kernel code.
%
% \begin{macro}{\renewcommandx}
% \begin{macro}{\xargs@renewc}
% The \xargs{}version of \cs{renewcommand}, and related internal macro.
%    \begin{macrocode}
\newcommand\renewcommandx{%
  \@star@or@long\xargs@renewc}
%    \end{macrocode}
%    \begin{macrocode}
\newcommand*\xargs@renewc[1]{%
  \begingroup\escapechar\m@ne
    \xdef\@gtempa{{\string#1}}%
  \endgroup
  \expandafter\@ifundefined\@gtempa{%
    \PackageError{xargs}{\noexpand#1undefined}{%
      Try typing \space <return> \space to proceed.\MessageBreak
      If that doesn't work, type \space X <return> \space to quit.}}%
    \relax
  \let\@ifdefinable\@rc@ifdefinable
  \xargs@newc#1}
%    \end{macrocode}
% \end{macro}
% \end{macro}
% \begin{macro}{\providecommandx}
% \begin{macro}{\xargs@providec}
% The \xargs{}version of \cs{providecommand}, and related internal macro.
%    \begin{macrocode}
\newcommand\providecommandx{%
  \@star@or@long\xargs@providec}
%    \end{macrocode}
%    \begin{macrocode}
\newcommand*\xargs@providec[1]{%
  \begingroup\escapechar\m@ne
    \xdef\@gtempa{{\string#1}}%
  \endgroup
  \expandafter\@ifundefined\@gtempa
    {\def\reserved@a{\xargs@newc#1}}%
    {\def\reserved@a{\renew@command\reserved@a}}%
  \reserved@a}
%    \end{macrocode}
% \end{macro}
% \end{macro}
% \begin{macro}{\DeclareRobustCommandx}
% \begin{macro}{\xargs@DRC}
% The \xargs{}version of \cs{DeclareRobustCommand}, and related internal macro.
%    \begin{macrocode}
\newcommand\DeclareRobustCcommandx{%
  \@star@or@long\xargs@DRC}
%    \end{macrocode}
%    \begin{macrocode}
\newcommand*\xargs@DRC[1]{%
  \ifx#1\@undefined\else\ifx#1\relax\else
    \PackageInfo{xargs}{Redefining \string#1}%
  \fi\fi
  \edef\reserved@a{\string#1}%
  \def\reserved@b{#1}%
  \edef\reserved@b{\expandafter\strip@prefix\meaning\reserved@b}%
  \edef#1{%
    \ifx\reserved@a\reserved@b
      \noexpand\x@protect
      \noexpand#1%
    \fi
      \noexpand\protect
      \expandafter\noexpand\csname
      \expandafter\@gobble\string#1 \endcsname}%
  \let\@ifdefinable\@rc@ifdefinable
  \expandafter\xargs@newc\csname
  \expandafter\@gobble\string#1 \endcsname}
%    \end{macrocode}
% \end{macro}
% \end{macro}
% \begin{macro}{\newenvironment}
% \begin{macro}{\xargs@newenv}
% \begin{macro}{\xargs@newenva}
% \begin{macro}{\xargs@newenvb}
% \begin{macro}{\xargs@new@env}
% The \xargs{}version of \cs{newenvironment}, and related internal
% macros.
%    \begin{macrocode}
\newcommand\newenvironmentx{%
  \@star@or@long\xargs@newenv}
%    \end{macrocode}
%    \begin{macrocode}
\newcommand*\xargs@newenv[1]{%
  \@testopt{\xargs@newenva#1}0}
%    \end{macrocode}
%    \begin{macrocode}
\@ifdefinable\xargs@newenva{%
  \def\xargs@newenva#1[#2]{%
    \kernel@ifnextchar[%]
      {\xargs@newenvb#1[#2]}%
      {\xargs@new@env{#1}{[#2]}}}}
%    \end{macrocode}
%    \begin{macrocode}
\@ifdefinable\xargs@newenvb{%
  \def\xargs@newenvb#1[#2][#3]{%
    \xargs@new@env{#1}{[#2][{#3}]}}}
%    \end{macrocode}
%    \begin{macrocode}
\newcommand\xargs@new@env[4]{%
  \@ifundefined{#1}{%
    \expandafter\let\csname#1\expandafter\endcsname
    \csname end#1\endcsname}%
    \relax
  \expandafter\xargs@newc
    \csname #1\endcsname#2{#3}%
  \l@ngrel@x\expandafter\def\csname end#1\endcsname{#4}}
%    \end{macrocode}
% \end{macro}
% \end{macro}
% \end{macro}
% \end{macro}
% \end{macro}
% \begin{macro}{\renewenvironment}
% \begin{macro}{\xargs@renewenv}
% The \xargs{}version of \cs{renewenvironment}, and related internal
% macro.
%    \begin{macrocode}
\newcommand\renewenvironmentx{%
  \@star@or@long\xargs@renewenv}
%    \end{macrocode}
%    \begin{macrocode}
\newcommand*\xargs@renewenv[1]{%
  \@ifundefined{#1}{%
    \PackageError{xargs}{\noexpand#1undefined}{%
      Try typing \space <return> \space to proceed.\MessageBreak
      If that doesn't work, type \space X <return> \space to quit.}}%
    \relax
  \expandafter\let\csname#1\endcsname\relax
  \expandafter\let\csname end#1\endcsname\relax
  \xargs@newenv{#1}}
%    \end{macrocode}
% \end{macro}
% \end{macro}
%
% \bigskip
% \begin{center}\Large
% That's all folks!\\
% Happy \TeX ing!
% \end{center}
%
% \Finale
